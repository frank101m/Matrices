%\def\largereport{1}
\ifdefined\largereport
	\documentclass[14pt,multi={minipage,math},preview,border=2cm]{standalone}
\else
	\documentclass[14pt]{article}
\fi


\usepackage[margin=0.4in]{geometry}
\usepackage{graphicx}
\usepackage{wrapfig}
\usepackage{amsmath}

\makeatletter
\renewcommand*\env@matrix[1][*\c@MaxMatrixCols c]{%
  \hskip -\arraycolsep
  \let\@ifnextchar\new@ifnextchar
  \array{#1}}
\makeatother

\newcommand\scalemath[2]{\scalebox{#1}{\mbox{\ensuremath{\displaystyle #2}}}}

\newsavebox{\fmbox}
\newenvironment{fmpage}[1]
{\begin{lrbox}{\fmbox}\begin{minipage}{#1}}
{\end{minipage}\end{lrbox}\fbox{\usebox{\fmbox}}}

\begin{document}

\begin{center}
\begin{minipage}{\linewidth}
\vspace{-0.5cm}
\begin{fmpage}{\linewidth}
\vspace{0.5cm}
\begin{wrapfigure}{r}{0.1\textwidth}
\vspace{-1.2cm}

\centering
\includegraphics[width=0.1\textwidth]{logo}
\end{wrapfigure}
\LARGE{\textbf {Reporte de operaciones con S.E.L}}

\vspace{0.5cm}
\large{Universidad Centroamericana ``Jos\'e Sime\'on Ca\~nas"} \\
\large{An\'alisis num\'erico}
\vspace{0.5cm}
\end{fmpage}
\vspace{1cm}

\large\textbf{{Reducci\'on gaussiana con sustituci\'on hacia atr\'as}} \\



\end{minipage}
\end{center}

\begin{math}
\\
\text{
Resoluci\'on del sistema de ecuaciones lineales A
} \newline \vspace{0.5cm} \newline
A = \quad
\gausssel
\end{math}
\vspace{0.5cm}
\begin{math}
\\
\text{
Proceso de reducci\'on gaussiana con la matriz aumentada $\mathbf{[A,B]} = \mathbf{\tilde{A}}^{(1)}$
} \\ \vspace{0.5cm} \\
\gaussmatrices
\\ \vspace{0.5cm} \\
\text{\Large{\textbf{Soluci\'on encontrada:}}} \newline \newline
\gausstable
\end{math}


\end{document}