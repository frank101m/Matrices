\documentclass{article}
\usepackage[margin=0.4in]{geometry}
\usepackage{graphicx}
\usepackage{wrapfig}
\usepackage{amsmath}

\newsavebox{\fmbox}
\newenvironment{fmpage}[1]
{\begin{lrbox}{\fmbox}\begin{minipage}{#1}}
{\end{minipage}\end{lrbox}\fbox{\usebox{\fmbox}}}

\begin{document}

\begin{center}
\begin{minipage}{\linewidth}
\vspace{-0.5cm}
\begin{fmpage}{\linewidth}
\vspace{0.5cm}
\begin{wrapfigure}{r}{0.1\textwidth}
\vspace{-1.2cm}

\centering
\includegraphics[width=0.1\textwidth]{logo}
\end{wrapfigure}
\LARGE{\textbf {Reporte de operaciones con S.E.L}}

\vspace{0.5cm}
\large{Universidad Centroamericana ``Jos\'e Sime\'on Ca\~nas"} \\
\large{An\'alisis num\'erico}
\vspace{0.5cm}
\end{fmpage}
\vspace{1cm}
\end{minipage}
\end{center}


%aca ira el body

\begin{center}
\large\textbf{{M\'etodo iterativo de Jacobi}}
\end{center}
Resoluci\'on del sistema de ecuaciones lineales A
\[
\jacobisel
\]

\[
\mathbf{A}=
\jacobimatrixA
%\begin{bmatrix}
%10&-1&2&0\\
%-1&11&-1&3\\
%2&-1&10&-1\\
%0&3&-1&8
%\end{bmatrix}
\]
F\'ormula del proceso iterativo:
\[
\mathbf{X}^{(k)} = \mathbf{TX}^{(k-1)} + \mathbf{C} \quad \forall \; k \geq 1
\]
Condiciones iniciales de proceso:
\[
\mathbf{T}=
\jacobimatrixT
%\begin{bmatrix}
%10&-1&2&0\\
%-1&11&-1&3\\
%2&-1&10&-1\\
%0&3&-1&8
%\end{bmatrix}
\quad
\mathbf{C}=
\jacobimatrixC
%\begin{bmatrix}
%1 \\
%1 \\
%1 \\
%1 
%\end{bmatrix}
\quad
\mathbf{X}^{(0)}=
\jacobimatrixXO
%\begin{bmatrix}
%0\\
%0\\
%0\\
%0
%\end{bmatrix}
\]
Resultados de apoximaciones del la soluci\'on $\mathbf{X} \approx \mathbf{X}^{(k)}$
\begin{center}
\jacobitables
\end{center}
\end{document}